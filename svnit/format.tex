\documentclass{exmppr}
\usepackage{tasks}
\usepackage{tikz}
\usepackage{epigraph}
\usepackage{enumerate}

\institutename{Sardar Vallabhbhai National Institute of Technology, Surat}
\sem{1}%which semester
\coursename{Algorithms and Computational Complexity}%name of course
\ccode{CO603}%course code
\type{MidSem}%endsem or midsemm
\season{Summer}%winter, fall, summer, etc.
\batch{2018-19}% batch 2013-14,2014-15,2015-16 etc

\profname{Udai P Rao}%name of prof

\newtheorem{exercise}{\bfseries}

\begin{document}
\begin{enumerate}

\item \begin{enumerate}
\item What are the certain criteria that algorithm has to follow? Discuss.

\item The Euclid's algorithm for finding the GCD establishes that sum of the values of variables will drop very very fast in number of iterations.\\
Comment on the above statement and discuss how number of iteration $\leq log_{3/2}(p+q)$? where p $\&$ q are input values to the algorithm.

\end{enumerate}
\item Let `n' be the number of files that can be stored on a sequential storage device such as tape. Let `$L_i$' be the length of each file where $1 \leq i \leq n$. Further, Mean Retrieval Time(MRT) is given by:\\
{\centering $$MRT = \sum_{i=1}^{n}\sum_{j=1}^{i} L_i$$}\\
According to difference ordering of files, MRT varies.
\begin{enumerate} 
\item How many ordering of files is/are possible?
\item Out of all possible ordering of files which ordering gives and optimal result, which will minimize MRT?
\item Provide an algorithm for this.
\item Provide example illustration and time complexity of the algorithm.
\end{enumerate}
\item Recurrence relation of finding maximum and minimum number for a given array of size n is given as below. 
Here, time T(n) is equal to the number of comparisons.\\
{\centering $$
  T(n) = \left \{
  \begin{aligned}
    &T(\lceil n/2 \rceil) + T(\lceil n/2 \rceil) + 2, n > 2 \\
    &1, n = 2\\
    &0, n = 1
  \end{aligned} \right.
$$}
\begin{enumerate}
\item Solve the given recurrence with back substitution method by taking n=2 as base case.
\item Above recurrence is based on divide and conquer approach . Comment on the space complexity of this algorithm in comparison to naive algorithm.
\end{enumerate}

\item \begin{enumerate}
\item Write down an algorithm for following statement:\\
You are given `n' activities with their start and finish times. Select the maximum number of activities that can be performed by a single person, assuming that a person can only work on a single activity at a time.
\item What is the time taken by the algorithm to execute?
\end{enumerate}

\item The Longest Increasing Subsequence(LIS) problem is to find the length of the longest subsequence of a given sequence such that all elements of the subsequence are sorted in increasing order. For example, the length of LIS for {10, 22, 9, 33, 21, 50, 41, 60, 80} is 6 and LIS is {10, 22, 33, 50, 60, 80}. Find the optimal substructure. Write and explain recurrence relation or recursive equation for given problem.\\
{\centering \textbf{OR}}
\\
Consider a variant of the matrix-chain multiplication problem in which the goal is to parenthesize the sequence of matrices so as to maximize, rather that minimize, the number of scalar multiplications. Does this problem exhibit optimal substructure? Justify your answer.

\item How many people should be invited to a party on Mars, where a year is 669 Martian days long, in order to make it likely that there are two people (one matching pair) with the same birthday Give an analysis to above problem using indicator random variables.

\item Using backtracking provide a solution for sum of subset problem. With m = 30 and w = {5, 10, 12, 13, 15, 18}, find all possible subset of w that sum to m. Draw state space tree for above problem.

\end{enumerate}

\newpage
\begin{center}\section*{Answers}\end{center}
%\input{sol1.tex}

\end{document}
